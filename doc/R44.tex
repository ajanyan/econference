\documentclass[10pt,a4paper,journal]{IEEEtran}
\usepackage{graphicx,subfigure}
%\usepackage[latin1]{inputenc}
\usepackage{amsmath}
\bibliographystyle{IEEEtran}
\usepackage[numbers]{natbib}
\renewcommand{\bibfont}{\normalsize}

\usepackage{mathptmx}
\usepackage{amsfonts}
\usepackage{amssymb}
\usepackage{makeidx}
\usepackage{url}
\usepackage{algorithm}
\bibliographystyle{apacite}
\usepackage{algorithmic}


\newcommand\Mark[1]{\textsuperscript#1}
\usepackage[T1]{fontenc}
\renewcommand{\figurename}{\bfseries\fontsize{10}{20}\selectfont \textbf{Figure } }


\linespread{1.1}
%\setlength{\columnsep}{.5em}
\usepackage[T1]{fontenc}
%Page Number
\usepackage{nopageno}
\usepackage[left=0.75cm,right=0.75cm,top=2cm,bottom=2cm]{geometry}
\setlength{\columnsep}{2.5em}
\title{FOG COMPUTING: Architecture, Characteristics and Applications}
\author{\IEEEauthorblockN{Sreerakhi V.\Mark{1},
Maya Mohan\Mark{2}, and Sruthy Manmadhan\Mark{3}}\\
\Mark{1}M.Tech First Year Student,
\Mark{2}\Mark{,}\Mark{3}Assistant Professor\\
\IEEEauthorblockA{Department of Computer Science and Engineering,\\
N.S.S College of Engineering, Palakkad \\
Email: \Mark{1}sreev1994@gmail.com,
\Mark{2}mayajeevan@gmail.com,
\Mark{3}sruthym.88@gmail.com }}


\begin{document}

\maketitle
\thispagestyle{plain}
\pagestyle{plain}
\begin{abstract}
Internet of Things (IoT) promises revolution that intensifies the calibre of day to day life, but generates abundant unprecedented data. Cloud computing is used as an efficient way to process data because of its high computation power and storage capability. This high volume of data is burdensome for the traditional system, the cloud. All the data and request need to be transmitted to the cloud so that the network bandwidth increases. This may lead to long latency. Most of the IoT application supports real-time applications, which make them unfit to the cloud architecture. To overcome the limitations of cloud computing a new technique is proposed, called fog computing which provide cloud like service. Fog computing is a technology that extends the abilities of the cloud to the network edge. Similar to cloud, fog provides data compute, storage and application services to the end devices. Unlike cloud computing, fog computing has decentralized architecture which ensures the proximity of the fog node to the end users. This paper describes the hierarchical architecture of fog computing and its characteristics, key technologies used, challenges and open issues. This paper also explains some application scenarios that can be successfully implemented using fog computing paradigm.

\end{abstract}
\begin{flushleft}
\begin{keywords}
\textit{\textbf{Keywords:}} Fog Computing, Cloud Computing, Internet of Things, Face identification, Smart Gateway, VANET
\end{keywords}
\end{flushleft}

\section{Introduction}
\hspace*{1em}Fog computing is a new technology proposed by CISCO to take full advantage of network edge devices. It extends the abilities of the cloud to the network edge. It is a distributed paradigm that provides cloud like services to the end users. It is developed to address the applications that do not fit in the paradigm of cloud \cite{1,2}. The most significant characteristic and the greatest advantage of fog computing is its proximity to the end users. The developers deploy fog application on to the access points basically routers, switches, etc.\\
\hspace*{2em} Fog infrastructure redistribute data and compute so that the much of the action takes place on the end devices. These data will be distributed according to how soon the action required on that data. Thus data can be categorised into three: \textit{real-time transitive data}, that kind of data is analysed on the fog node that are closer to the devices generating the data; \textit{data that can wait for seconds or minutes} for processing, this kind of data is send to the aggregation node and after processing send it back to the fog node so that the device will act on that data; and \textit{less time sensitive data}, this kind of data is directly pushed into the cloud.\\
\hspace*{2em}	 Fog computing keeps data right where the Internet of Things need it, so that the data transfer time and the amount of data movement across the internet are greatly reduced. The data that are generated by sensors and devices are collected, processed and stored by the fog node at the network edge. It provides speedy and high quality services, hence by enable low latency. Thus it meets the requirements of real time services. Instead of transmitting whole the data over internet, the end devices send only the useful data that is in need to be transmitted to cloud. Thus it will save the band width of the network. The network edge devices and the users are connected to fog mainly through wireless communication mode such as Bluetooth, WiFi and 4G.\\
\hspace*{2em}	Cloud computing is used as an efficient way to process data because of its high computation power and storage capability. What we do in this traditional system is that, we take this data and pushed into the cloud. So that most of the computational tasks such as data processing, filtering, cleaning, redundancy removing, etc needed to be takes place at the cloud. This need not be happened. For huge amount of data the network bandwidth is becoming the bottle neck of cloud computing. It leads high latency. The other limitation of cloud computing includes security shortcoming. Encryption was failed in securing the data from the attacker. It does not verify whether the user is authorised or not. Nobody is getting identified when the attack happens. Also it is complex to detect which user is attacked. Fog computing could help by solving most of these limitation of the cloud computing.\\

\section{RELATED WORKS}

\hspace*{1em}The fog computing technology deals with IoT data locally by utilizing clients or network edge devices close to the users to carry out a high degree of storage, management, control, communication, and configuration. Fog computing involves the components of data processing and analysis in distributed system.
       
 \begin{figure}
	\centering
	\includegraphics[scale=0.5]{fog-arch.jpg} 
	\caption {The hierarchical architecture of fog computing \cite{2}}
\label{Fig 1}
 \end{figure}
    
\subsection{Hierarchical Architecture}
\hspace*{1em}	The hierarchical architecture of fog computing is illustrated in Figure \ref{Fig 1}. The architecture consists of three layers \cite{2}. The layer which is closest to the end devices is the end user is called \textit{terminal layer}. This layer consists of various IoT devices for example sensors, smart vehicles, smart cards, mobile phones, etc. The devices in this layer will sense data from the environment and send to its upper layer for processing using wireless communication modes.\\
\hspace*{1em}	The second layer is the \textit{fog layer} which is located on the edge of the network. This layer contains widely distributed large number of fog nodes including routers, switches, base stations, etc. This layer is able to compute, process and store received data. Only a less amount of use full data is transmitted to the top most layers, the cloud layer for further processing or storage. The fog layer is connected to cloud use wired or wireless connection. \textit{Cloud layer} consists of several high performance servers and storage devices and provides various application services, such as smart home, smart transportation, smart factory, etc.

\subsubsection{Key technologies}
	Computing technology includes computation offloading and latency management. Fog computing offloads part of tasks to nearby node. Also it overcomes resource constraints on edge devices. This technology helps with improving performance and battery life time. The main objective of latency management is to limit the ultimate service response time with in an acceptable threshold.
	The common wireless communication technologies include 3G, 4G, WiFi, WLAN, Zigbee and Bluetooth. Some other communication technologies are the following: 

\begin{itemize}
\item	Software Defined Network (SDN): It helps with efficient management of heterogeneous fog networks. It can solve issues such as irregular connectivity, collision and high packet loss.
\item Network Function Virtualization (NFV): The network function is separated from the dedicated physical network hardware to get the maximum advantages of virtualization and device abstraction technology. 
\item The 5th Generation (5G) Wireless Communication System: The advantages of 5G include wide signal coverage, high network speed, high flux density and high mobility. 
\item Content Distribution Network (CDN): It can help with less bandwidth usage, reduced network congestion, higher content availability and reduced cost.
\item Long Reach Passive Optical Network (LRPON): To extend network reach up to 100 Km, with large number of optical network units. It supports latency-sensitive and bandwidth intensive applications such as smart home, smart industry services.

\end{itemize}

\subsubsection{Storage Technologies} This technology is to meet the demand of low latency property in fog computing, the pre-cache technology can be considered for the same. The delay on downloading contents from cloud can be reduced by caching most desirable contents on the demand of users. Hence make full use of storage devices to provide most desirable services to users.

\subsubsection{The Comparison between Cloud and Fog Computing}
	Fog and cloud computation modes have their own benefits for establishing the needs of computing
tasks in corresponding scenarios \cite{3}. The comparison is shown in Table I.
\begin{center}
\begin{table}[htbp]
	
	\caption[Comparison between Cloud and Fog]{Comparison between Cloud and Fog [2]}

\begin{tabular}{p{2.5cm}  p{2.5cm}  p{3cm} }
\hline
 & \textbf{Cloud Computing} & \textbf{Fog computing}\\
\hline 
\textbf{Latency} & High & Low\\

\textbf{Real time interactions} & Supported & Supported\\

\textbf{Mobility} & Limited & Supported\\

\textbf{Location awareness} & Partially supported & Supported\\

\textbf{Number of server nodes} & Few & Large\\

\textbf{Geographical distribution} & Centralized & Decentralized and distributed\\

\textbf{Distance to end devices} & Far (multiple hops)& Near (single or few hops)\\

\textbf{Location of service} & Within the Internet & At the edge of the local network\\

\textbf{Working environment} & Specific data center building with AC systems & Outdoor(streets,etc.) or indoor(houses, cafes)\\

\textbf{Communication mode} & IP network & Wireless communication: WLAN, 4G, etc. or wired communication\\

\textbf{Dependence on the quality of core network} &	Strong & Weak\\
\textbf{Bandwidth costs}	& High & Low\\

\textbf{Computation and storage capabilities} & Strong & Weak\\

\textbf{Energy consumption} & High & Low\\
\hline
\end{tabular}
\end{table}
\end{center}

\subsubsection{Challenges and Open Issues}
	Realizing fog computing's full potential presents several challenges including balancing load distribution between edge and cloud resources, API and service management and sharing, and SDN communications. Since fog computing devices are distributed in various environments, with no strict protection, various system security problems may happen. The security solution that applies to cloud computing may not work since fog devices work at the network edge \cite{4}. Some important attack against fog computing are listed below.
\begin{itemize}
\item Man in the Middle: MITM attack happens when a communication between two systems is intercepted by an outside entity. Secure communication protocol cannot be implemented in fog devices.
\item Distributed denial of service (DDOS): Fog nodes are resource constraint. So it is very difficult to handle large number of request simultaneously. Fog node become very busy for a long time on the arrival of number of irrelevant requests.
\end{itemize}

\hspace*{1em}	In fog environments, resource management systems should be able to dynamically determine which analytics tasks are being pushed to which cloud- or edge-based resource to minimize latency and maximize throughput. These systems also must consider other criteria such as various countries' data privacy laws involving, for example, medical and financial information.
	Constructing a real IoT environment as a test bed for evaluating such techniques is costly and doesn't provide a controllable environment for conducting repeatable experiments. Also scheduling tasks in fog computing is complex compared with cloud computing. A fog computing application is typically spread over the client's mobile device, one of potentially many fog nodes, and occasionally a back-end cloud server. Therefore, deciding where to schedule computational tasks in fog computing is more difficult \cite{5}.

\subsection{Smart E-Health Gateways} 
\hspace*{1em}	The majority of medical devices, for example wearable and implantable medical sensors generate data, but they are not capable of storing it \cite{6}. If there are large numbers of connected devices the latency of the connection with the cloud could be significant. More over these devices are bandwidth constrained. This make them unfit to the cloud architecture. Here a more responsive design is fog computing. Components of IoT based health monitoring system is shown in Figure \ref{Fig 2}. 

\begin{figure}
  \centering
	\includegraphics[scale=0.5]{e-hosp-arch-gr3.jpg}
	\caption[The components of fog based health monitoring system ]{The components of fog based health monitoring system   \cite{7}}
\label{Fig 2} 
\end{figure}

\subsubsection{System Workflow}
	The network of various sensors read data, which can be health related information of a patient as well as the current environmental conditions \cite{7}. The sensor network uses three types of sensors: the medical sensors for monitoring a patient's vital information such as ECG value, body temperature, blood pressure, SPO2, respiratory rate and heat rate; environmental sensors to measure environmental light, humidity and temperature; and finally the activity sensors to monitor motion of a patient for example steps, posture and activity. 
	UT-GATE collects the data from all the sensors via UDP server at the gateway on port 5700. The collected data will pass through filters first for noise reduction. Band pass filter with Finite Impulse Rate (FIR) is implemented in the gateway for filtering. Then the gateway applies data fusion on the filtered sensory data. For further noise reduction and avoid its impact on the process, anomaly detection method will be takes place. The vital information of the patient, collected by the medical sensors is used to calculate Early Warning Score (EWS). EWS can be used to analyse the health condition of a patient. When EWS value increases a notification is sent to indicate the condition is critical. The data from the activity sensors are used for fall detection. In this case also a notification is send when the patient is tending to fall. XML format is used to implement notification messages and is sent by an android application communicating through WiFi.
	The processed data will be compressed to take a backup so that data will be safe even in the lost internet connection. Also in order to fulfil latency requirement LZW algorithm is used, which is a loss less data compression method and is efficient even in higher input size. To implement local storage UT-GATE uses MYSQL database. The data will be made secure by an Asymmetric Encryption Method using Crypto Library in Python. UT-GATE powered by Ubuntu operating system uses a firewall called Uncomplicated Firewall (UFW) to restrict the accessibility of protocol and ports. Finally whenever there is an internet connection the data will be transmitted to the cloud for permanent storage and further analysis.

\subsection{Face Identification and Resolution Scheme}
\hspace*{1em}	The face identification system takes a critical part in distinguishing people characters and represents their identity. In the traditional approach what we do is we just capture the facial image of an individual and sent it to the cloud from where face detection, face image pre-processing, etc. up to face identifier generation takes place  \cite{8}. It is not suitable to use centralized processing architecture if the number of applications and access users are too high. 
\subsubsection{Face Resolution Process}
	Fog computing and cloud computing work together to complete the resolution process as shown in Figure \ref{Fig 3}. The fog computing based face resolution process is explained as follows.

\begin{figure}
  \centering
	\includegraphics[scale=0.15]{face-res.jpg}
	\caption {The fog computing based face resolution framework  \cite{8}}
\label{Fig 3} 
\end{figure}

\begin{itemize}

\item An individual's facial images are captured by the visual sensors and send it to the client. The client receives this requests FN for face resolution services. After establishing proper connection client send this images to the FNs with in the fog layer.

\item Face detection, facial image pre-processing, feature extraction and face identifier generation are executed at the FN to get the face identifier. Face identifier resolution service is requested by FN to cloud. After establishing network connection successfully, FNs send the generated face identifier to MS in the cloud.

\item To execute face resolution process, MS receives and sends the face identifier to RS.

\item RS receives the face identifier and executes two operations. Firstly, the extracted facial identifier is matched with the face identifiers stored in data center. Secondly, RS gets the URI address of corresponding identity information that is paired with the matched face identifier. At last RS returns the URI address to MS.

\item To collect the individual's elaborated identity information MS connects with IS using URI address.

\item IS returns the elaborated identity information to MS.

\item MS transmits the elaborated identity information to FNs.

\item FNs transmit the individual's elaborated identity information to the client, and then display it to the end users.

\end{itemize}

\subsection{Framework for Process monitoring and Prognosis in Cyber manufacturing}
\hspace*{1em}	The objective of fog-based cyber-manufacturing systems is to provide the foundation to next-generation smart manufacturing networks in which manufacturers will have access to on-demand computing infrastructures, mobile applications for cyber-manufacturing and parallel machine learning tools \cite{9}. Fog computing provides multiple high speed processors to compliment large storage and speedy I/O operations. This automatically increases the computing capacity of the system \cite{10}. Existing process monitoring system and prognosis methods have limited capabilities of collecting and storing large volume of data \cite{11}. 
\subsubsection{Work flow}

The Figure \ref{Fig 4} illustrates the work flow for fog based online machine and process monitoring and prognosis. Each step is described in detail below:

\textbf{Step1:} Collect machine information by gateway devices. An interoperable information acquisition gateway device collects real-time streaming information from industrial plant floors through sensing element networks, communication adapters, sensor adapters, and I/O adapters. These adapters are developed using communications protocols like Simple Object Access Protocol (SOAP), MTConnect, and Open Platform Communications Unified Architecture (OPC UA).

\textbf{Step 2:} All the raw datasets and sample datasets are streamed to a local private edge cloud and a remote public HPC cloud, respectively. All the real-time datasets are collected and streamed into a local private edge cloud, through the data acquisition gateway devices. Training datasets are streamed into a remote public cloud. These sample datasets are used to train predictive models using machine learning algorithms.

\textbf{Step 3:} Diagnostic and prognostic models are developed on the basis of training datasets. using cloud-based distributed/parallel machine learning algorithms. The key benefits of cloud-based machine learning algorithms over conventional machine learning techniques is that the sanctioning of large-scale machine learning through parallel implementation on the cloud. Whereas Hadoop MapReduce is an efficient approach for processing huge amounts of informations, it might incur a significant runtime cost for computational-intensive workloads such as highly-iterative machine learning algorithms. Apache Spark runs programs significantly faster than Hadoop MapReduce both in memory and on disk because Spark supports cyclic information flow and in-memory computing.

\textbf{Step 4:} The diagnostic and prognostic models will be applied to the raw datasets and stored in the local private edge cloud for online diagnosis and prognosis. The diagnostic and prognostic models developed on the remote public cloud using the training datasets are applied to real-time raw streaming datasets stored on the local private edge cloud. 

\begin{figure}
  \centering
	\includegraphics[scale=0.33]{cyber-manu-frmwrk.png}
	\caption {Computational frame work for Fog based cyber manufacturing system  \cite{9}}
\label{Fig 4} 
\end{figure}

\subsection{Vehicular Ad-hoc networks (VANET)}
\hspace*{1em}	VANET (Vehicle Ad-hoc Networks) is emergent technologies that they deserve, recently, the attention of the industry and the academic institutions \cite{12}. The VANETs support large variety of applications. In this rather than moving indiscriminately, vehicles tend to move in an organized fashion. Existing VANET applications are based on Vehicular Cloud Computing (VCC). Compared to VCC, fog computing has ability to support applications with low latency. Currently, this is no definitions of vehicular fog computing networks. This chapter just discuss some application scenarios based on fog computing in VANETs. 


\subsubsection{Smart traffic lights and connected vehicles}
	In this scenario the traffic signals getting changed by the vehicle and location. Here smart lights serve as fog device to send warning signal to approaching vehicles. The communication between vehicles and fog devices are through WiFi, 3G, road side units and smart traffic lights \cite{13}. Also neighbouring smart lights interacts with each other to share information.
\subsubsection{Software Defined Networks (SDN)}
	SDN is a growing computing and networking concept. SDN concept together with fog computing will resolve the main issues in vehicular networks irregular connectivity, collisions and high packet loss rate. SDN supports vehicle to vehicle with vehicle to infrastructure communications and main control \cite{13}.
	SDN support vehicle-to-vehicle(V2V) as well as infrastructure-to-vehicle(I2V) communication. Hence SDN based architecture provides flexibility, scalability, programmability and global knowledge. Developers proposed a new VANET architecture by combining SDN and fog computing paradigm called Software Defined Networking based VANET (FSDN VANET).
\subsubsection{Parking System}
	The traffic is basically in an exceedingly mess once the quantity of vehicles is increasing quickly \cite{14}. As a consequence, finding a parking space is remarkably troublesome and overpriced. This scenario focuses on finding parking problem to alleviate the traffic congestion, cut back pollution and enhance driving effectively in the view of IoT. Fog computing and roadside cloud are utilized to find a vacant spot.
\subsubsection{Decision Support Systems}
	Developers proposed DSS is for driver safety and traffic violation monitoring \cite{15}. Now it is just a conceptual framework, but they believes that it can be easily adapted to the current scenario for future hassle free driving rule violation monitoring system.



\section{DISCUSSION AND FINDINGS}
\hspace*{1em}	In IoT, fog computing will provide unified interfaces and versatile resources to accomplish heterogeneous procedure and storage requests. Fog computing keeps data right where the Internet of Things need it, so that the data transfer time and the amount of data movement across the internet are greatly reduced.The data that are generated by sensors and devices are collected, processed and stored by the fog node at the network edge. 
%It provides speedy and high quality services, hence by enable low latency. Thus it meets the requirements of real time services. 
Instead of transmitting whole the data over internet, the end devices send only the useful data that is in need to be transmitted to cloud. Thus it will save the band width of the network. The network edge devices and the users are connected to fog mainly through wireless communication mode such as Bluetooth, WiFi and 4G. Computing technology includes computation offloading and latency management. In order to solve issues such as irregular connectivity, collision and high packet loss SDN is implemented on fog. The network function is separated from the dedicated physical network hardware to get the maximum advantages of virtualization and device abstraction technology through NFV. 5G wireless communication system also included in addition to 3G and 4G in order to provide wide signal coverage, high network speed, high flux density and high mobility. CDN can help with less bandwidth usage, reduced network congestion. To extend network reach up to 100 Km, with large number of optical network units LRPON is used. The delay on downloading contents from cloud can be reduced by caching most desirable contents on the demand of users.\\	
\hspace*{2em}	Fog computing might be helpful in healthcare, within which real-time operation and event response are crucial. One proposed system utilizes fog computing to discover, predict, and stop falls by stroke patients. Proposed fog computing based smart-healthcare system allows low latency, mobility support, and location and privacy awareness compared to ancient system. The developed Smart e-Health Gateways takes the responsibility of data processing (data filtering, data compression, data fusion and data analysis), adaptability, local storage, security, interoperability, reconfigurability, device discover, mobility support, energy efficiency for sensor nodes and low latency.\\
\hspace*{2em}	A fog computing-based face identification and resolution scheme to solve the matter of computation, communication and storage facing in giant scale accessing to cloud resolution services. Within the current system all the information processing and identifier generation takes place at the cloud. It is modified within the proposed system, wherever face identifier is generated at the fog layer; only the identifier is transmitted to cloud. This framework can even be appropriate for many of different biometrics.\\
\hspace*{2em}	Existing process observation system and prognosis strategies have restricted capabilities of collecting, grouping and storing massive volume of knowledge. Also it has restricted the real-time computational capability. Fog computing based framework for process observation and prognosis in cyber manufacturing will with success solve these problems. Instead of transferring all data from one server to another or from one place to a another, the manufacturers will collect and analyse massive volumes of information at a local network edge device. This will reduce latency. Fog computing provides manufacturers with secure, scalable, reliable and feasible storage for large static and dynamic knowledge. Manufactures can store data locally, without sharing to the cloud.\\
\hspace*{2em}	Vehicular Fog Computing(VFC) employs vehicles as the infrastructures to create the most effective use of those vehicular computational and communication resources. VFC establishes extremely virtualized communication and computing facilities at the proximity of mobile vehicles in VANET. Existing VANET applications are based on Vehicular Cloud Computing (VCC). Compared to VCC, fog computing has ability to support applications with low latency. Comparison between Conventional method and Fog based method of each application scenarios are shown in Table II.

\begin{center}
\begin{table}
	\centering
	\caption[Comparing Conventional and Fog based methods]{Comparing Conventional and Fog based methods}
\begin{tabular}{p{0.5cm}  p{2cm}  p{2.5cm}  p{2.5cm} }
\hline
\textbf{Paper} & \textbf{Application} & \textbf{Conventional model} & \textbf{Fog based model}\\
\hline
\\
  \cite{7}& Smart e-Health gateways at the edge of healthcare internet &  Whole data need to be transferred to cloud from where all data processing, actuation and storage are executed. 
&  Local data processing, Local storage, Local actuation are executed at fog layer. \\
\\
\cite{8} & Face identification and resolution scheme & Face identifier is generated at cloud. & Face identifier generated at fog layer and transmitted to cloud. \\
\\
\cite{9}& Framework for process monitoring and prognosis in cyber manufacturing  & Prognostic methods have limited capability  of collecting and storing large amount of data, limited computational capacity & Process monitoring, Diagnosis, Prognosis  at fog layer.  Capable to deal with large amount of data. Support real-time services.\\
\\
\cite{13}& Vehicular Ad-hoc networks & Based on Vehicular Cloud Computing (VCC), limited latency, limited support for mobility. & Edge location awareness, Geographical distribution, Real time interactions, Support for mobility\\
\\
\hline

\end{tabular}
\end{table}
\end{center}
		


\section{CONCLUSION}
\hspace*{1em}	Fog computing could be a high-potential computing model whose significance is growing chop-chop because of the quick development of IoT and Mobile Internet. It permits the seamless integration of edge and cloud resources. It greatly reduces the information transfer time and the amount of network transmission, and effectively meet the strain of real-time or latency sensitive applications and ease network bandwidth bottlenecks. The architecture of Fog Computing, key technologies, applications, challenges and open issues are summarized and surveyed in intimately. The concept of fog computing and Smart e-Health Gateways within the context of Internet-of-Things based healthcare systems was bestowed. Fog computing-based face identification and resolution scheme to unravel the matter of computation, communication and storage facing in giant scale accessing to cloud resolution services was bestowed. Fog computing-based framework for data-driven machine health and process observation in cyber manufacturing has been bestowed. Finally several interesting application scenarios, challenge issues of fog computing in VANETs have been identified and discussed.


\begin{thebibliography}{15}
	
\bibitem{1}    R. Want, B. N. Schilit, and S. Jenson, 2015, "Enabling the Internet of Things", \textit{Computer}, Volume 48, pp. 28-35

\bibitem{2}  PengfeiHu, Sahraoui Dhelim, Huansheng Ning, Tie Qiu, 2017, "Survey on fog computing: architecture, key technologies, applications and open issues", \textit{Journal of Network and Computer Applications}, Volume 98, pp. 27-42

\bibitem{3}  Amir Vahid Dastjerdi, Rajkumar Buyya, 2016, "Fog Computing: Helping the Internet of Things Realize Its Potential", \textit{Computer}, Volume 49, Issue 8, pp. 112-116

\bibitem{4} S. Yi, Z. Qin, Q. Li, 2015 , "Security and privacy issues of fog computing: A Survey", \textit{Wireless Algorithms, Systems, and Applications}, pp. 685-695

\bibitem{5} Zijiang Hao, Ed Novak,  Shanhe Yi, Qun Li, 2017, "Challenges and Software Architecture for Fog Computing", \textit{IEEE Internet Computing}, Volume 21, Issue 2, pp. 44-53

\bibitem{6} B. Farahani, F. Firouzi, V. Chang, M. Badaroglu, and K. Mankodiya, 2017, "Towards Fog-Driven IoT eHealth: Promises and Challenges of IoT in Medicine and Healthcare", \textit{Future Generation Computer Systems}, Volume 78, pp. 659-676

\bibitem{7} Amir M. Rahmani, Tuan Nguyen Gia, Behailu Negash,Arman Anzanpour, Iman Azimi, Mingzhe Jiang, Pasi Liljeberg, 2017, "Exploiting smart e-Health gateways at the edge of healthcare Internet-of-Things: A fog computing approach", \textit{Future Generation Computer Systems}, Volume 78, Part 2, pp. 641-658 

\bibitem{8} Pengfei Hu, Huansheng Ning, Tie Qiu, Yanfei Zhang, Xiong Luo, 2017, "Fog Computing Based Face Identification and Resolution Scheme in Internet of Things", \textit{IEEE Transactions on Industrial Informatics}, Volume 13, Issue 4, pp. 1910-1920

\bibitem{9} Dazhong Wu, Shaopeng Liu, Li Zhang, Janis Terpenny, Robert X. Gao,Thomas Kurfess, Judith A. Guzzo, 2017, "A fog computing-based framework for process monitoring and prognosis in cyber-manufacturing", \textit{Journal of Manufacturing Systems}, Volume 43, Part 1, pp. 25-34

\bibitem{10} H. Ning, H. Liu, J. Ma, L. T. Yang, and R. Huang, 2016, "Cybermatics: Cyberphysical-social-thinking hyperspace based science and technology", \textit{Future Generation Computer Systems}, Volume 56, pp. 504-522

\bibitem{11} Wu D, Rosen DW, Wang L, Schaefer D, 2015 , "Cloud-based design and manufacturing: a new paradigm in digital manufacturing and design innovation", \textit{Computer Aided Design}, Volume 59, pp. 1-14

\bibitem{12} Yang Fangchun, Wang Shangguang, Li Jinglin, Liu Zhihan, Sun Qibo, 2014, "An overview of Internet of vehicles", \textit{China Communications}, Volume 11, Issue 10, pp. 1-15

\bibitem{13} Kang Kai, Wang Cong, Luo Tao, 2016, "Fog computing for vehicular Ad-hoc networks: paradigms, scenarios, and issues", \textit{The Journal of China Universities of Posts and Telecommunications}, Volume 23, Issue 2, pp. 56-65

\bibitem{14} Jian Liua, Jiangtao Li, Lei Zhang, Feifei Dai, Yuanfei Zhang a, Xinyu Meng, Jian Shenb, 2018, "Secure intelligent traffic light control using fog computing", \textit{Future Generation Computer Systems}, Volume 78, pp. 817-824

\bibitem{15} Dua A, Kumar N, Bawa S., 2014, "A systematic review on routing protocols for vehicular ad hoc networks",  Vehicular Communications, Volume 1, Issue 1, pp. 33-52

\end{thebibliography}



% \bibliographystyle{plain}
% 
%\bibliography{dna}
\end{document}
