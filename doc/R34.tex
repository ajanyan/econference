\documentclass[10pt,a4paper,journal]{IEEEtran}
\usepackage{graphicx,subfigure}
%\usepackage[latin1]{inputenc}
\usepackage{amsmath}
\bibliographystyle{IEEEtran}
\usepackage[numbers]{natbib}
\renewcommand{\bibfont}{\normalsize}

\usepackage{mathptmx}
\usepackage{amsfonts}
\usepackage{amssymb}
\usepackage{makeidx}
\usepackage{url}
\usepackage{algorithm}
\bibliographystyle{apacite}
\usepackage{algorithmic}



\usepackage[T1]{fontenc}
\renewcommand{\figurename}{\bfseries\fontsize{10}{20}\selectfont \textbf{Figure } }


\linespread{1.1}
%\setlength{\columnsep}{.5em}
\usepackage[T1]{fontenc}
%Page Number
\usepackage{nopageno}
\usepackage[left=0.75cm,right=0.75cm,top=2cm,bottom=2cm]{geometry}
\setlength{\columnsep}{2.5em}
\title{SMART CITY ARCHITECTURE}
\author{\IEEEauthorblockN{Athira S$^1$,
Maya Mohan$^2$, and Sruthy Manmadhan$^3$}\\
$^1$M.Tech First Year Student,
$^{2,3}$Assistant Professor\\
\IEEEauthorblockA{Department of Computer Science and Engineering,\\
N.S.S College of Engineering, Palakkad \\
Email: $^1$athi231295@gmail.com,
$^2$}mayajeevan@gmail.com,
$^3$sruthym.88@gmail.com }}
%\markboth{Department of Computer Science \& Engineerinng,~December,~2013}{}

%{Shell
%\MakeLowercase{\textit{et al.}}: A Novel Tin
%Can Link}

\begin{document}

\maketitle
\thispagestyle{plain}
\pagestyle{plain}
\begin{abstract}
A smart city is an urban area that uses different types of electronic data collection/sensors to supply information used to manage assets and resources efficiently.he smart city concept integrates information and communication technology (ICT), and various physical devices connected to the network (the Internet of things or IoT) to optimize the efficiency of city operations and services and connect to citizens.The paper discusses different architectures introduced to automatize the various aspects of a smart city.
\\
\textit{Keywords}-Smart City, Ethics Aware, Cloud, integrated Architecture, Big Data Analytic.


\end{abstract}


\section{INTRODUCTION}
\hspace{2em}A city is a relatively large area with complex system for transport, utilities, sanitation, decision making etc.A smart city is a novel paradigm introduced to simplify the complex systems to remotely monitor, manage and coordinate the devices.Smart city initiatives my include sensors being deployed to everything and connecting them to the internet inorder to achieve intelligent recognition, tracking ,location etc.Inorder to achieve enhanced quality, performance and interacivity of the urban services normally Information and communication technology (ICT) are used .They may also reduce costs and resource consumption and  increase contact between citizens and government.Smart city applications are developed to manage urban flows and allow for real-time responses.A smart city may therefore be more prepared to respond to challenges than one with a simple "transactional" relationship with its citizens.\\

\hspace{2em}The most important term that is related to an smart city is Internet of Things\cite[1].The Internet of things (IoT) is the network of smart devices embedded with sensors (RFID \cite{2}, IR, GPS, laser scaners),actuator and other network connectivities that enables the objects to collect the data and exchange the data using the underlying internet framework available.Each object is uniquely identifies by the computer system using  Internet infrastructure.The IoT allows data to be collected remotely from different sensors and communicate with each other in such a way that they enable real word scenario to be represented as computerized systems  thus resulting in improved efficiency, accuracy ,economic benefit and reduced human intervention.When IoT is augmented with sensors and actuators,the technology represents a general class of cyber physical system,which encompasses technologies such as smart homes, intelligent transportation and smart cities.\\

\hspace{2em}Another terms related to the same are Big Data and Cloud.Big data is a term that are used to represent large set of data for which the processing may be complex than the traditional data processing .Challenges in processing Big Data includes  capturing heterogeneous data, data storage, data analysis, search, sharing, transfer, visualization, querying, updating and information privacy.Cloud computing generally refers to a network of remote servers that are hosted on the internet rather than some local server or some personal computers.Thus cloud computer enable individuals to store data remotely over the internet thus increasing the availability of data.


\section{RELATED WORKS}
\\
\hspace{2em}The paper summarizes some smart city architectures.

\subsection{ETHICS AWARE ARCHITECTURE}

\hspace{2em}Various architectures has been proposed for energy efficient technology but none concerns with the main aspect of a smart city i.e, people.The Ethics Aware Object Oriented Smart City architecture(EOSCA)\cite{3} deals with including traditions, culture and laws into the Smart city initiative.
\hspace{2em}The architecture include two distinguished features.Firstly, an object oriented architecture that describes real world things as objects with some ethics parameters.Secondly, introducing an ethics later to indulge socio cultural and ethics into these object.\\

\hspace{2em}The architecture defines smart city as follows:\\

\hspace{2em}\textit{An interdisciplinary approach, leveraging advanced computing and communication technologies, to make intelligent use of public resources in order to improve quality of services as well as quality of life offered to citizens, such that it fosters holistic well being of society.}\cite{3}\\

\hspace{2em}The most challenging task is to encode ethics into machine because of its fuzzy nature.A fully fledged moral agents known as artificial moral agents\cite{4} are used to incorporate ethics to machine.The architecture incorporates ethics, moral,cultural, regional parameters(called Ethics of Operation or EOP)relevant to the context to form ethics aware things i.e, objects with some ethic based values.First the EOP used to form propositional statements.Based on these statements various scenarios of interaction are identified.Then the ethics response for each these scenarios are designed in the form of Boolean equations.Finally the responses are implemented as hardware or software.\\

\hspace{2em}EOSCA \cite{3} proposes a layered architectures with five layers: abstraction layer, network layer, data analytic layer, ethics layer and business layer.\\
\begin{figure}[htbp]
\centering
\includegraphics[scale=.4]{ethics.png}
\caption{Ethics-Aware Object-Oriented Smart City Architecture(EOSCA)\cite{3}}
\label{1}
\end{figure}
\\
\subsubsection{ABSTRACTION LAYER}
A smart city might require to gather highly private, sensitive and strategic data from different sources via sensors, smart phones and other embedded devices.Mere identification of data and collection may not be adequate, the gathered data should be relevant to the particular context.The layer describes objects that represent real world things in a particular format with ethics parameters.Describing objects other than data enables to have better control over the smart things and it makes it easy to aggregate the data for analytics.The object has two characteristics features one that it incorporate privacy and security parameters and second,it include ethics in the scope of an object.
\\
\subsubsection{NETWORK LAYER}
Communication technologies.The short distance M2M communication uses technologies like RFID \cite{2}, near field communication \cite{5} etc. For point to point communication involving small data transfer Z-waves are used.The layer exhibits ethical behaviour by employing QoS attributes to the objects.
\\
\subsubsection{DATA ANALYTICS LAYER}
Since data are collected from enormous number of devices it may result in large amount of data being collected. The data analytic layer provides a cloud computing platform to store and manage huge data.The layer aggregates object on a particular context into compound object.Personal informations are not considered in this layer . The compound objects may represent intangible concepts like  climatic conditions, agricultural health, crime rate etc.Data mining from the objects are non trivial process that may be used to retrieve valuable informations.The layer also ensures security and privacy\cite{6} in smart city.
\\
\subsubsection{ETHICS LAYER}
The most important part of a smart city is its people .Data may be collected from different sources that may lead to social problems of inequality.Different societies may vary in their moral,ethical and cultural values so a smart city may vary as well.The role of ethics layer is to form limits in the form of code, rules based on the  values of the society to form ethic based decisions.The layer makes sure that the data collected from different sources are handled in such a manner that they 
conforms social, cultural and legal standards.The basic building blocks of this layer comprises of simple and compound object to ensure ethical decision making.
\\
\subsubsection{BUSINESS LAYER}
Smart city initiatives provide various market opportunities to the manufacturers, service providers and business man.The layer ensures tat there are no ethical concerns regarding the services provided to the citizens.Business layer only permit the transactions that do not violate the social cultural and traditional standards.The layer gathers input from data analytic and ethics layer to understand the dynamics in the economy.
\\
\subsection{THE CLOUD BASED ARCHITECTURE}
\\
\hspace{2em}}Homeland management is one of the most relevant application of smart city that is being attracted by various authorities.When ever a crisis event occurs in an area the authorities are responsible of taking adequate measures.The main challenge to deal with is real time responses based on the information available.Hybrid cloud architecture\cite{7} is used for controlling, managing and storing resources needed to command and control the activities in an emergency scenario.In an extremely integrated and collaborative scenario the emergency crisis management may be an useful service provided by smart cities.\\
\hspace{2em}Along with the wireless and ubiquitous network that enable communication using sensors another important paradigm is cloud computing that provide extremely scalable and run time storage environment for heterogeneous data.The challenges involved in providing city wide services include collecting, integrating, aggregating and processing enormous amount of data from different sources.The emergency management scenario may include geographically distributed sensors called Mobile Emergency Operation Center(MEOC) that may detect any variation from the usual scenario.The activities of MEOC are controlled and commanded by the Command Emergency Operation Center(CEOC).\\
\hspace{2em}In an specific emergency scenario may include various buildings, sites etc and more than one crisis may occur at the same time. Thus the extension and complexity of the existing network JAN,IAN AND PAN and the amount of resources that may be required for the management of the crisis may not be known before head.Because of this uncertainty the computing infrastructure that is used in the emergency crisis may include store retrieve and runtime processing of radio maps that may support rapid recovery in case of failures in the mobile sensing devices.In addition each individual node comprises of limited amount of private resources to deal with specific crisis.The hybrid cloud paradigm is a good architecture that guarantee some level of flexibility in managing MEOC and CEOC runtime and storage services.\\

\begin{figure}[htbp]
\centering
\includegraphics[scale=.5]{C.png}
\caption{The layered hybrid cloud architecture.\cite{7}}
\label{1}
\end{figure}
\hspace{2em}The cloud based architecture is based on three main layer "Infrastructure as a Service(IaaS)", "Platform as a Service (PaaS)" and "Software as a Service (SaaS)" each specialized for particular goals.\\
\hspace{2em}The lowest layer is the hybrid Iaas that integrates private and public computing and storage resources.The IaaS layer guarantees the harmonic coexistence of hybrid resources in transparent and secure environment.It may be achieved by extending LAN  networks across private and public resources.\\
\hspace{2em}The PaaS layer is used to provide specialized runtime functions for particular scenarios.They consist of MEOC deployed at mobile nodes with adequate resources and processing for tat particular nodes.And CEOC with large database for the storage of enormous amount of data and processing capacity that are commanded and managed centrally.It also contains a Emergency Orchestrator that are used to manage and control the entire crisis event.During emergency management the Orchestrator is responsible for handling the resource  usage and automatically expanding and reducing them based on some cloud auto scaling paradigm.\\
\hspace{2em}The last level SaaS are used to implement these MEOC and CEOC services.It guarantee resilience and fail over by underlying federation cloud infrastructure.The components are directly used by the on scene emergency operators in case of failure.\\

\begin{figure}[htbp]
\centering
\includegraphics[scale=.4]{C1.png}
\caption{The cloud infrastructure in the emergency management scenario\cite{7}}
\label{2}
\end{figure}
\\
\subsubsection{Network Virtualization}
The network virtualization allows flexible orchestrator of the resource that are available in different domains supporting the connection of various MEOC and CEOC with specific emergency management contexts.The architecture framework provide effective VPN as a service solutions that facilitate vertical and horizontal communication between the private and public cloud and the sensing devices.\\

\subsection{INTEGRATED ARCHITECTURE }

\hspace{2em}In an smart city data may be collected from enormous number of heterogeneous sources.Thus the data needs to be processed in real time and delivered to the applications used in smart cities.The integrated architecture describes the flow of data from source to the end users.\\
\hspace{2em}More and more applications today use, generate and handle large amount of data.The data may be of different format and storage hence makes it difficult to handle context data.Considering the rapid growth of data, extraction of valuable information from raw data is a challenging task especially when the data are from heterogeneous sources.\\
\hspace{2em}The given architecture\cite{8} performs big data processing from source to destination in seven steps:collection of data from heterogeneous sources, data normalization, data brokering, data storage, data analysis, data visualization and decision support system.\\
\hspace{2em}Data may be aggregated from different sources, then normalization of data is performed .Before doing it data from personal devices are anonymized.Now the context of the collected data is generated and send to storage in parallel and distributed way.Then the data is analysed to generate patterns and discover sights.Later the data are visualized for efficient decision making.\\
\begin{figure}[htbp]
\centering
\includegraphics[scale=.3]{I.png}
\caption{The data flow architecture\cite{8}}
\label{2}
\end{figure}
\hspace{2em}The first step is to collect data from different sources via smart sensors, personal devices, batch data from city services etc inorder to extract valuable knowledge from it.Also data relevant to a particular context are combined together to empower decision making.\\
\hspace{2em}Once the data is collected it needs to be made sure that the data id anonymous and cannot be tracked back using the available data.Thus data anonymization module is used to provide privacy to the data so as to avoid data leakage.Then this data is normalized.\\
\hspace{2em}After normalizing the data  reaches context broker.The context broker divides the given piece of data and put hem into relevant context.A context broker is a service that gather data of different type, sources and velocity and then based in certain condition the data are integrated together to a particular context.\\
\hspace{2em}Now links are created between the data and the context information and send it to the distributed storage and processing framework.Now the data patterns are extracted from the batch data at real time data processing.\\
\hspace{2em}The Hadoop framework are used for batch processing and storm.Hadoop Distributed File Systems are used to store and process the data.The big advantages of using Hadoop are scalability, fault tolerance and guaranteed data processing.\\
\hspace{2em}While processing of data big data analytics, statistical and numerical analysis are used to gain valuable assets.In the end, the data are visualized to enable value added decision making.
  \\
\subsection{BIG DATA ANALYTIC}
\hspace{2em}With the introduction of smart things the conventional methods are driven back.Introduction to IoT has made revolutionary changes in the smart city architecture.This architecture uses Big data analytics for real time data processing and intelligent decision making.Also the architecture is integrated with WoT for accessibility  and manageability of smart devices over the internet standards.The architecture is generally used for energy management in smart buildings.\\
\hspace{2em}Big data analytics architecture\cite{9} has four layers 
\begin{enumerate}
\item Data creation and collection level
\item Data processing and management level
\item Event and decision management level
\item Application level
\end{enumerate}
\hspace{2em}The framework for smart city aims to reduce the unnecessary consumption and traffic congestion of the city.The data are collected from operational tasks in day to day life.But  acquiring data from all the operational task is a challenge.Thus large number of low cost and energy efficient sensors are deployed within the city area and connected to smart devices that are responsible for data collection.Home appliances are equipped with ZigBee sensors to read real time energy consumption and are connected over the net to control smart home remotely over the web.Data collected from this layer are transferred to the data processing level for determining value added informations.\\
\hspace{2em}The architecture uses multiple modalities such as HDFS \cite{10}, HBase and HIVE\cite{11} inorder to facilitate data storage and processing.HDFS act as a storage medium under Hadoop framework.\\
\begin{figure}[htbp]
\centering
\includegraphics[scale=1]{BD.png}
\caption{Big data embedded smart city integrated with RESTFUL WoT\cite{9}}
\label{2}
\end{figure}
\hspace{2em}The architecture uses two nodes Hadoop cluster for better performance.Hbase is used to speed up the processing as they act as speedy real time lookups, in memory caching and server side programming and they also prove fault tolerance.Hive is used for querying and management of enormous amount of data.Upon storage the control is transferred to the event and decision management layer.\\
\hspace{2em}The event and decision management layer performs both event as well as decision management functionality.The event management classifies the processed information into two two events, the service and the resource events.Service event controls the services of the smart city and he resource event manages the resources of smart city.The decision management generates intelligent decision corresponding to some service event.\\
\hspace{2em}Application layer connects the smart gateways integrated with WoT to allow remote access to the smart city services.All the smart city services corresponding to service and resource events.The layer is again divided into department layer , service layer and sub service layer.\\

\begin{figure}[htbp]
\centering
\includegraphics[scale=.37]{BD1.png}
\caption{Sub layering at the application level for event generation and processing\cite{9}}
\label{2}
\end{figure}

\hspace{2em}The department layer is responsible for intelligent decision making.This layer ascertains high and low priority decision and unicast the decision to the upper layers.Service layer transfers the corresponding decisions to the sub service layer components.The sub service layer generates respective smart city service to serve citizens.\\
\section{DISCUSSIONS AND FINDINGS}
\hspace{2em}The Ethics aware object oriented smart city architecture\cite{3} incorporates the most important part of a city that is the people.With the change in regions the ethical values may get altered thus we need to consider different architecture based on the changes in the culture ,ethics etc.EOSCA deals with this also providing some privacy and security to the data.It also enhances the marketability and business values.Yet The major draw back in this architecture is that the transformation of ethics into statements are difficult to implement and rather costly.\\
\hspace{2em}The cloud based architecture for emergency management\cite{7} provides real time responses in case of the emergency scenarios.Cloud platform is used as a effective storage of data thus providing a large number of application.Even though cloud act as a effective storage implementation of advanced city services are all the most difficult.Also real time processing of large number of simultaneous request is a challenging task.\\
\hspace{2em}Integrated architecture \cite{8} provides a seven step architecture for data processing that is efficient, fault tolerant and scalable yet the processing does not include any Data reduction, data cleaning or optimization of pattern extraction.\\
\hspace{2em}Big Data analytics based Smart city Architecture\cite{9} mainly focuses on providing  efficient home energy management.They also provide high processing speed and throughput.The difficulty with this architecture is that it is not platform compatible and the provide Non ubiquitous access to urban services.\\


\section{CONCLUSION}
\hspace{2em}The paper summarises several smart city architectures that are introduced for the development of various aspects of the smart city.Ethics architecture incorporates ethics along with cloud to make ethics based decisions in a smart city.The cloud based architecture used for emergency responses in case of emergency situations.The integrated architecture provides a seven layer data processing mechanism. And the Big data analytics architecture provide energy management in smart buildings.
\begin{thebibliography}{8}
	
\bibitem{1} Tai-hoon Kim, Carlos Ramos \& Sabah Mohammed,(2017), \textit{\enquote "Smart City and IoT.}Future Generation Computer Systems"},Vol: 76, PP 159-162. ELSEVIER.
\bibitem{2}  D.L.Yang, F. Liu, \& Y.D. Liang,(2010),  \textit{\enquote "A Survey of the Internet of Things"}, in Proceedings of the 1st International Conference on E-Business Intelligence (ICEBI2010)}, pp. 358–366
\bibitem{3}Sahil Sholla, Roohie Naaz, \& Mohammad Ahsan Chishti China Communications,(2017), \textit{\enquote " Ethics Aware Object Oriented Smart City Architecture.}IEEE Journals & Magazines"} , Vol: 14, Issue: 5, PP: 160 - 173.IEEE.
\bibitem{4} W. Wallach and C. Allen,(2009),\textit{\enquote " Moral machines : teaching robots right from wrong."} Oxford University Press .}
\bibitem{5} R. Want,(2011),\textit{\enquote “Near field communication,”} IEEE Pervasive Computer}, vol. 10, pp. 4–7.
\bibitem{6} Z. K. Zhang, M. C. Y. Cho, C. W. Wang, C. W. Hsu, C. K. Chen, \& S. Shieh,(2014),\textit{\enquote “IoT security: Ongoing challenges and research opportunities,” }in Proceedings - IEEE 7th International Conference on Service-Oriented Computing and Applications,} SOCA 2014, pp. 230–234.
\bibitem{7}Francesco Palmieri , Massimo Ficco , Silvio Pardi \& Aniello Castiglione,( 2016 ),\textit{\enquote “A cloud-based architecture for emergency management and first responders localization in smart city environment." } Computers & Electrical Engineering},Vol:56,  PP:810-830.ELSEVIER
\bibitem{8}Cristian Chilipirea, Andreea-Cristina Petre, Loredana-Marsilia Groza,Ciprian Dobre \& Florin Pop,(2017),\textit{\enquote " An integrated architecture for future studies in data processing for smart cities." }Microprocessor and microsystems},VoL:52, PP: 335-342.ELSEVIER
\bibitem{9}Bhagya Nathali Silva, Murad Khan \& Kijun Han,(2017),\textit{\enquote "Integration of Big Data analytics embedded smart city architecture with RESTful web of things for efficient service provision and energy management." }Future Generation Computer Systems(ACCEPTED)}.ELSEVIER
\bibitem{10}Nawab Muhammad Faseeh Qureshi, Dong Ryeol Shin, Isma Farah Siddiqui, Bhawani Shankar Chowdhry,(2017),\textit{\enquote "Storage-Tag-Aware Scheduler for Hadoop Cluster"},IEEE Journals & Magazines},Vol:5,PP: 13742 - 13755.
\bibitem{11}Xin Chem, Liting Hu, Liangqi Liu, Jing Chang, Diana Leante Bone,(2017),\textit{\enquote " Breaking Down Hadoop Distributed File Systems Data Analytics Tools: Apache Hive vs. Apache Pig vs. Pivotal HWAQ"},IEEE Conference Publications,}Pages: 794 - 797.
\bibitem{12}A. Al-Fuqaha, M. Guizani, M. Mohammadi, M. Aledhari \& M. Ayyash, (2015).\textit{\enquote "Internet of Things: A Survey on Enabling Technologies, Protocols, and Applications," }IEEE Communications Surveys & Tutorials,} vol. 17, no. 4, pp. 2347 - 2376.
\bibitem{13} J. Pan, R. Jain, S. Paul, T. Vu, A. Saifullah \& M. Sha, (2015), \textit{\enquote "An Internet of Things Framework for Smart Energy in Buildings: Designs, Prototype, and Experiments,"}IEEE Internet of Things Journal }, vol. 2, no. 6, pp. 527 - 537. 
\bibitem{14}A. Zanella, N. Bui, A. Castellani, L. Vangelista \& M. Zorzz, (2014),\textit{\enquote  "Internet of Things for Smart Cities,"} IEEE Internet of Things Journal}, vol. 1, no. 1, pp. 22-32.
\bibitem{15}Angelo Cenedese, Andrea Zanella and Lorenzo Vangelista ,Padova ,(june 2014),\textit{\enquote "SmartCity: An urban internet of things experimentation in:2014 IEEE 15th International Symposium on World of Wireless, Mobile and Multimedia Networks}}.
\end{thebibliography}
	




%\hspace{2em} Through this paper  wish to discuss with the importance of Cognitive Computing and introduces an model for the Cognitive Computing and a Simple simulator for the same.
%
%\hspace*{2em}
%
%\hspace*{2em} COMPASS is simulator to implement CC in application level.\cite{1} It helps to implement brain like functions in a hardware platform. COMPASS if fully based on the architecture TrueNorth, a non-von Neumann, modular,
%parallel, distributed, event-driven, scalable architecture inspired by the function, low power, and compact volume
%of the organic brain.
%
%\hspace*{2em} The biological cells in the brain called neurons have been modelled for for different mathematical functions.So our aim is to capture input-output behaviour of neuron
%and try to implement it using mathematical models.
%\section{Applications \cite{3}} 
%\begin{enumerate}
%\item Speaker Recognition 
%\item Composer Recognition
%\item Digit Recognition
%\item HMM Sequence Modeling
%\item Collision Avoidance
%\item Optical Flow
%\item Eye Detection
%\end{enumerate}


\end{document}
